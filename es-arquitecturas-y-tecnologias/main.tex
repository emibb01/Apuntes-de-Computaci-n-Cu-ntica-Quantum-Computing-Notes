\documentclass[pdflatex,sn-mathphys-num]{sn-jnl}
\usepackage{graphicx}%
\usepackage{physics}
\usepackage{multirow}
\usepackage{amsmath,amssymb,amsfonts}
\usepackage{amsthm}
\usepackage{mathrsfs}%
\usepackage[title]{appendix}%
\usepackage{xcolor}%
\usepackage{textcomp}%
\usepackage{manyfoot}%
\usepackage{booktabs}%
\usepackage{algorithm}%
\usepackage{algorithmicx}%
\usepackage{algpseudocode}%
\usepackage{listings}%
\usepackage{comment}
\usepackage{natbib}
%% as per the requirement new theorem styles can be included as shown below
\theoremstyle{thmstyleone}%
\newtheorem{theorem}{Theorem}%  meant for continuous numbers
\newtheorem{proposition}[theorem]{Proposition}% 

\theoremstyle{thmstyletwo}%
\newtheorem{example}{Example}%
\newtheorem{remark}{Remark}%

\theoremstyle{thmstylethree}%
\newtheorem{definition}{Definition}%



\raggedbottom
%%\unnumbered% uncomment this for unnumbered level heads

\begin{document}

\title[Article Title]{Notas para la clase de computación cuántica.}

\author*[1,2]{\fnm{Barragán} \sur{Emiliano}}\email{emilianob@ciemcias.unam.mx}


\maketitle

El campo de la computación cuántica(QC) y de la información cuántica(QI) es uno de los campos de investigación activa más prometedores de los últimos 45 años. El trabajo de Feyman y otras personas nos introdujeron por primera vez un nuevo lenguaje de computación, basado en la física cuántica, que nos permitía abandonar la lógica binaria de su contraparte clásica, proveyéndonos una manera novedosa para simular nuevos problemas y diseñar nuevos algoritmos con la meta final de realizar ciertas tareas de manera más rápida que su contraparte clásica.\\

A pesar de que la QC y la QI son dos de las áreas de investigación más vanguardistas, estas utilizan uno de los sistemas cuánticos más sencillos, el modelo de dos niveles. Esto tiene una enorme ventaja desde el punto de vista didáctico, ya que permite desarrollar actividades interesantes y útiles incluso en un entorno de pregrado. No obstante, el QC y la QI tienen potencial para aplicaciones en muchas áreas diversas de las ciencias físicas y sociales, como la criptografía (procedimientos de comunicación más seguros utilizando principios cuánticos), finanzas (algoritmos de optimización cuántica para guiar el comercio), la gravedad cuántica(Simulaciones más exactas de agujeros negros utilizando circuitos cuánticos), entre otras.\\

A pesar de los avances significativos en los algoritmos cuánticos y la teoría de la comunicación, el mayor impedimento en el campo es la creación de una computadora cuántica escalable con la capacidad de realizar las tareas donde teóricamente debería tener ventaja a una computadora clásica. En los últimos años ha habido algunos avances en la construcción de estas computadoras; sin embargo, aún no existe un consenso científico sobre las plataformas más adecuadas para construir los qubits y la mejor arquitectura para las computadoras cuánticas, por eso estás notas buscan dar una introducción de alguno de los métodos utilizados a la fecha de la escritura de estas notas.

Las notas se organizan de la siguiente manera: en la primera parte se presentará y analizará las ventajas y desventajas de alguno de los métodos utilizados para crear qubits, mientras que en la segunda sección se abordarán las dificultades actuales para crear una arquitectura para una computadora cuántica así como uno de los problemas más grandes en el área que es la mitigación del error.

% Main content
\section{Algunas Tecnologías para la creación qubits.}

Se explicará de manera breve algunas de las maneras de crear qubits que se utilizan en la actualidad y se analizará su viabilidad para crear Computadoras Cuánticas, para esto se usará los criterios de DiVincenzo:

\begin{itemize}
    \item Los qubits deben de estar bien caracterizados y ser fáciles de producir. Además de ser escalables i.e. deben poder manejar una cantidad arbitraria de qubit.
    \item Debe ser posible iniciar un qubit en un estado base e.g |0>
    \item El tiempo de coherencia debe ser mayor al de típicos cálculos.
    \item Debe ser posible implementar un grupo universal de compuertas.
    \item Debe existir una manera confiable de leer la información de un qubit.
    
\end{itemize}

\subsection{Iones atrapados}\label{sec2}
Un ion es un núcleo o molécula a la que se le ha aumentado o disminuido su nube de electrones por una unidad. Después de ionizar un elemento, este presentará un comportamiento hidrogenoide, con estados de energía 
$\ket{a_{i}}$ con $i\epsilon\mathbf{N}$ caracterizados por niveles de energía $E_{i}$. Dos de estos estados serán caracterizados y definidos como $\ket{1}$ y $\ket{0}$.
Tomando en cuenta el movimiento del ion en el espacio, su Hamiltoniano tomaría la forma:
\begin{equation}
    H=H_{a}+\frac{\Vec{P}}{2m}+V(x)
\end{equation}
Un ion recién creado, por ejemplo $^{40}Ca^{+}$, tiene una velocidad $\approx$ de 1m/s, a tales energías se puede considerar como un sistema clásico de "motional prespective", para crear un sistema cuántico debemos limitar el ion en rango (atraparlo) y en energía(enfriarlo)\\
Las dos trampas más usadas son la trampa de Penning y la trampa de Paul.En este artículo solo se analizará de manera superficial la trampa de Paul, ya que esta es la más utilizada en ambientes de academia e industria.\\
Un teorema planteado por Earnshaw nos dice que una partícula cargada no puede encontrar equilibrio en potenciales eléctricamente estáticos que satisfagan la ecuación de Laplace. Sin embargo, es posible demostrar que en el caso de potenciales rotando a frecuencias particulares aparecen soluciones de la ecuación de movimiento correspondiente a un punto de equilibrio dinámico, logrando así confinar la partícula en un volumen.\\
Suficientemente, cerca del centro del volumen donde confinamos nuestro ion, al cual llamamos nuestra trampa, podemos aproximar el comportamiento del ion como un oscilador armónico en 3 dimensiones. El sistema más utilizado de esta trampa en C.C. es conocida como "ion trap", donde la fuerza de dos dimensiones es mucho mayor al de la última. Para que esta aproximación se cumpla necesitamos que el ion se mantenga cerca del centro de la trampa. Por esta razón es necesario "enfriar al ion"(en este contexto enfriar se refiere a disminuir la ENERGÍA CINÉTICA) para ello se utiliza un método conocido como "laser cooling", y por último, para evitar la colisión entre partículas, ya que estas colisiones causan cambios en su energía cinética se pondrá en un estado de vacío ultra-alto, esto se traduce a que en la región en la que está nuestra trampa la presión es del orden $10^{-11}Torr$.
\subsubsection{Láser cooling}
El enfriamiento se divide en dos regímenes, uno clásico, en el que el valor esperado $\expval{\hat{H_{SHO}}}$ es mucho mayor que el valor del oscilador cuántico simple(QSHO), $\hbar\omega_{z}$ y uno cuántico, donde $\expval{\hat{H_{SHO}}}$ es comparable a $\hbar\omega_{z}$
\subsubsubsection{Doppler cooling(Enfriamiento en el régimen clásico)}
Un ion se encuentra normalmente en el régimen clásico cuando es creado, el método más común usado para el enfriamiento en este régimen es conocido como "Dopler Cooling". Se comienza identificando dos estados, teniendo identificados la transición entre dos estrados con frecuencia $\omega$ y "linewidth" $\Gamma$, que en este contexto el "linewidth" representa el ancho máximo de absorción de fotones en el espacio de frecuencia. Donde $\Gamma$ es mucho mayor que $\hbar\omega_{z}$, a estos iones se les irradia con un láser monocromático con frecuencia un poco menor al de la transición entre los estados y momento $\hbar\Vec{k}$.Ahora suponemos un ion que absorbe un fotón y, por lo tanto, se excita y de manera casi inmediata emite un fotón regresando a su estado base.

\begin{figure}
    \centering
    \includegraphics[width=0.5\linewidth]{imagenes//iones/efecto_doppler.png}
    \caption{El inciso a) y b) representa el ion absorbiendo fotones del láser monocromático dirigiéndose en su dirección y el inciso c) es la emisión en una dirección aleatoria del fotón. El ancho de las flechas representa la cantidad de momento de los fotones o del ion según corresponda. Imagen tomada de \cite{Bernardini_trapped_ions}}
    \label{fig:efecto-dopples}
\end{figure}
Comenzando en el momento en el que el ion absorbe el fotón del láser.
Las ecuaciones de conservación de la energía del sistema es:
\begin{equation}\label{clasical_energy_cons}
    \hbar\omega_{0}+\frac{1}{2}Mv^{\prime 2}=\hbar\omega_{abs}+\frac{1}{2}Mv^{2}
\end{equation}
Y la de la conservación del momento del sistema es
\begin{equation}\label{clasical_momentum_cons}
   Mv^{\prime}=Mv+\hbar\Vec{k}
\end{equation}
Donde los términos con subíndice abs representan los valores después de la absorción del fotón. Sustituyendo el valor de $v^{\prime}$ de la ecuación (\ref{clasical_momentum_cons}) en la ecuación (\ref{clasical_energy_cons}) obtenemos:\
\begin{equation}
    \omega_{0}=\omega_{abs}-v\cdot\Vec{k}-\frac{\hbar\Vec{k^{2}}}{2m}
\end{equation}
El segundo término de esta ecuación es conocido como el "cambio Doppler", y el tercero como el "cambio de recoil" que es mucho menor al cambio Doppler a altas velocidades, por lo que despreciaremos el término de recoil para estos primeros cálculos.\\
Ahora pasamos al proceso por el cual el ion emite el fotón, esta vez la combinación de la ecuación de la conservación del momento y la energía, nos da:
\begin{equation}
    \omega_{0}=\omega_{em}-v\cdot\Vec{k}+\frac{\hbar\Vec{k^{2}}}{2m}
\end{equation}
Promediando sobre todos los procesos de "scattering" por los que se emite el fotón, $\expval{\hbar v^{\prime}\cdot\vec{k}}=0$ por lo que el cambio promedio en la energía por el scattering es:
\begin{equation}
    \hbar\nabla\omega=\expval{\hbar(\omega_{em}-\omega_{abs})}=-\hbar v\cdot\Vec{k}
\end{equation}
Esta ganancia de energía fotónica $v\cdot\Vec{k}$ es igual a la perdida de energía cinética del ion i.e. $\nabla E_{k}=\hbar v\cdot\Vec{k}+\hbar<0$, llevando al "enfriamiento".\\
Este enfriamiento no es ilimitado, una forma de encontrar el límite es tomando en cuenta que cuando el ion se enfría lo suficiente, el término de recoil ya no puede ser despreciado. Tomando esto en cuenta el cambio en la energía cinética se vuelve 
\begin{equation}
    \nabla E_{k}=\hbar v\cdot\Vec{k}+\hbar k^{2}/2M<0\Rightarrow -\hbar v\cdot\Vec{k}>\hbar k^{2}/2M
\end{equation}
con $v\cdot\Vec{k}<0$, imponiendo así un límite hasta para el cual podemos usar este procedimiento para el enfriamiento, este punto es conocido como el límite de recoil.

\subsubsection{Régimen cuántico(Side band cooling).}

Después del Doppler cooling el ion se encuentra lo suficientemente frío para considerar los grados de libertad del QSHO, en este punto el ion tiene en promedio $\expval{n}=10$ niveles de energía. Para iniciar nuestro tratamiento del qbit, debemos enfriarlo al estado base, para esto se elige un "linewidth" $\Gamma_{s}$, tal que $\Gamma_{s}<<\omega_{z}$ para que después de la absorción se pueda diferenciar estados $\omega_{0}\pm\omega_{z}$, de los estados de frecuencia $\omega_{0}$. Llamamos el estado menor y mayor de esta transición, el estado base $\braket{g}$ y el estado excitado $\braket{e}$. El hamiltoniano libre de este sistema es entonces:

\begin{equation}
    H_{0}=\hbar\omega_{z}a^{\dagger}a+\frac{\hbar}{2}\omega_{0}\sigma_{z}
\end{equation}
Inicialmente, el ion se encuentra en el estado $\braket{g,n}$, donde esta notación nos indica que se encuentra en un estado atómico $\braket{g}$ y estado de vibración QSHO $\braket{n}$. La meta es conseguir encontrar el estado $\braket{g,0}$, esto lo conseguimos haciendo las siguientes transiciones
\begin{equation}
    \braket{g.n}\rightarrow\braket{e,n-1}\rightarrow\braket{g,n-1}\rightarrow\braket{e,n-2}\rightarrow...
\end{equation}
Este conjunto de transiciones las conseguimos al hacer interactuar el ion con un láser. El hamiltoniano de esta interacción está dado por:
\begin{equation}
    H=\frac{1}{2}\hbar\Omega(\sigma_{+}+\sigma_{-})\left[e^{i(kz-\omega t+\phi)}+e^{-i(kz-\omega t+\phi)}\right]
\end{equation}
Donde $k$, $\omega$ y $\phi$ es el vector de onda, la frecuencia y la fase del láser. $\Omega$ es la fuerza de interacción entre iones y $\sigma_{+}$, $\sigma_{-}$ son los operadores de subida y bajada para los grados de libertad internos del sistema.\\
Definimos $\eta=k\sqrt{\frac{\hbar}{2M\omega_{z}}}$ como el parámetro de Lamb-Dicke. Tras alguna aproximación en las que no profundizaremos, pero se pueden ver en profundidad en \cite{Bernardini_trapped_ions}
,podemos elegir la frecuencia de disparo del láser $\nabla=\omega-\omega_{0}=-\omega_{z}$, de la aproximación obtenemos el Hamiltoniano independiente del tiempo de la resonancia.
\begin{equation}
    H^{rsb}\approx\frac{1}{2}\hbar\eta\Omega(a\sigma_{+}e^{i\hat{\phi}}+a^{\dagger}\sigma_{-}e^{-i\hat{\phi}})
\end{equation}

Entonces, el operador de evolución-temporal es:
\begin{equation}
    U^{rsb}(t)\braket{g,n}=\cos{\Bar{\omega}t}\braket{e,n-1}
\end{equation}
Donde $\Bar{\omega}=\eta\Omega\sqrt{n}/2$. Como consecuencia inmediata de esto, podemos controlar la evolución del operador. 
Si aplicamos el láser por $t=\pi/\eta\Omega\sqrt{n}$ desaparece el término del coseno y el término de seno toma el valor de $\braket{e,n-1}$, inmediatamente el ion decae en $\braket{g,n-1}$. Haciendo esto repetidamente podemos llegar al estado base del QSHO.
\subsubsection{Análisis de las ventajas y desventajas.}

Como vimos en las secciones anteriores, los qbits creados con los iones pueden ser llevados al estado base con gran certeza y utilizando láseres lo suficientemente precisos, esto nos permite caracteriazar y aizlar de manera efectiva los dos niveles de energia base del ion, por lo tanto, cumple el primer y segundo criterio de Divincenzo.

En cuanto al tercer punto, el tiempo de coherencia de los iones de alrededor de $10^{-1}s$, que es mucho mayor al tiempo promedio necesario para aplicar una compuerta ($10^{-6}$s)\\
Con la información dada hasta el momento no podemos asegurar que se cumpla el criterio 4 y 5, para más información sobre la manipulación y lectura de los qubits se puede leer %referencia.
. Los criterios anteriores se cumplen sin problema para uno o un número pequeño de qbits. Lo que nos lleva al mayor problema de los iones atrapados es la escalabilidad.
El tiempo de coherencia de larga cadenas de iones se acorta, esto es un gran problema para cadenas de 100 o más iones, donde la coherencia es en general menor al necesario para aplicar las compuertas. Leer la información para estas cadenas también se vuelve un problema, ya que no existe láseres lo suficientemente precisos para obtener de manera confiable la información que contienen.\\
En la actualidad se está trabajando en nuevas formas de manipular los iones atrapados para mejorar su escalabilidad, para más información se pueden leer los artículos %ref 74 y 75

\subsection{Átomos neutros}

Otros de los métodos más comunes para la creación de qbits y una de las tecnologías que avanzan más rápidamente son los átomos neutros, utilizado en la industria por empresas como QuEra. Los átomos neutros, como indica su nombre, son aquellos donde el número de electrones y protones es el mismo. 
Los átomos neutros pueden ser utilizados como qbits explotando los niveles de energía "hyperfine", este término se refiere a pequeños cambios y separaciones en los niveles de energía atómica, y los estados de Rydberg, estados excitados que siguen la fórmula de Ridberg, la cual será desarrollada más adelante.\\
Profundizando en la estrctura hyperfine, esta estructura es relevante en sistemas de muy bajas temperatuuras. Sea $\mathbf{S}$ el operador spin electronico y $\mathbf{I}$ el operador Spin Nuclear, el Hamiltoniano de la estrcutura hyperfine esta definido como:
\begin{equation}
    H=\lambda\sum_{k}\mathbf{S_{k}}\otimes\mathbf{I_{k}}
\end{equation}
Que abrebiamos como $\lambda\mathbf{S}\cdot\mathbf{I}$\\
Definimos el operador spin hyperfine como:

\begin{equation}
    \mathbf{F}\equiv\mathbf{S}\otimes I+I\otimes\mathbf{I}
\end{equation}
De igual manera introducimos la manera mas sencilla de escribir el operador como $\mathbf{F}=\mathbf{S}+\mathbf{I}$.\\
Considerando la identidad

\begin{equation}
    \mathbf{S}\cdot\mathbf{I}=\frac{1}{2}\left[F(F+1)-I(I+1)-3/4\right]
\end{equation}
Derivado de $F^{2}$ 

\subsubsection{Trampa Magneto-Óptica}

Las trampas magneto ópticas son una de las formas más comunes de atrapar átomos neutros en temperaturas muy bajas.\\
Supongamos un átomo neutro ficticio, con niveles de energía mostrados en la figura \ref{fig:niveles-de-energia-atomo-neutro}

\begin{figure}
    \centering
    \includegraphics[width=0.5\linewidth]{imagenes//atomos-neutros/niveles-de-energia-atomos-neutros.png}
    \caption{Estados excitados con $J^{\prime}=1$ separación-Zeeman por un gradiente de un campo magnético variando en $z$. Si el átomo se mueve en dirección $-z$ del centro de la trampa, absorbe un fotón polarizado $\sigma^{+}$ llevando a la transición $m_{J}=0\rightarrow m_{J^{\prime}}=1$ y como resultado es empujado al centro de la trampa.}
    \label{fig:niveles-de-energia-atomo-neutro}
\end{figure}

el cual es colocado al centro de dos bobinas anti-helmholtz, para generar un campo magnético y seis láseres polarizados linealmente en frecuencia de rojos rodeándolo, tal que tres tiene polarización positiva $\sigma^{+}$ y tres $\sigma^{-}$, colocados como en la figura. 

\begin{figure}
    \centering
    \includegraphics[width=0.5\linewidth]{imagenes//atomos-neutros/trampa-magneto-optica.png}
    \caption{Ejemplo de una trampa magneto-óptica}
    \label{fig:enter-label}
\end{figure}

Sometemos al átomo a un campo magnético dependiente del espacio inducido por las bobinas $B(z)=\frac{dB_{z}}{dz}(-\frac{x}{2},\frac{y}{2},z)$. Los estados excitados $J=1$ sufren una separación-Zeeman tal que la energía del estado excitado $\braket{Jm_{J}}$ se separa como
\begin{equation}
    \Delta E=m_{J},g_{J},\mu_{B},\frac{dB_{z}}{dz}(-\frac{x}{2},\frac{y}{2},z)
\end{equation}
A este punto suponemos que nuestro átomo ha sido previamente enfriado por los métodos ya discutidos en la sección iones atrapados(\ref{sec2}), entonces nuestro átomo tendera a atrapar una mayor cantidad de átomos positivos $\sigma^{+}$ al moverse hacia uno de los ejes positivos y hará lo contrario al moverse hacia uno de los ejes negativos, esto empujara nuestro átomo hacia el centro de la trampa, consiguiendo lo deseado. Es importante notar que para hacer más pequeña la trampa se debe disminuir el gradiente del campo magnético, mientras se mantiene el "linewidth" del láser.
\subsubsection{Ventajas y desventajas}
El método de átomos neutros es fácilmente escalable, superando en ese sentido a los iones atrapados. Átomos neutros previamente enfriados pueden ser atrapados por una MOT de hasta $\approx10^{6}$ sin perder tiempo de coherencia, por lo que cumplen y superan a los iones atrapados en el primer criterio de DiVincenzo.\\
En cuanto al segundo criterio, los átomos neutros, al igual que los iones atrapados, son enfriados utilizando laser cooling, por lo que cumplen de igual manera con el segundo criterio de DiVincenzo. En cuanto a su tiempo de coherencia, es esperado que sea mayor al de sus compuertas y por su carga neutra, este tiempo no debería aumentar para larga cadenas, siendo esta su mayor ventaja referente a los iones atrapados.\\
En cuanto las compuertas estas pueden ser aplicadas sin ningún problema para un solo qbit, siendo estás más difíciles de aplicar para un par de ellos, siendo esta su mayor desventaja, ya que aún no hay maneras fiables y rápidas de aplicar compuertas a pares de qbits para átomos neutros y por su carga neutra la aplicación de las compuertas es más lenta que en otros métodos. Su medición, en cambio, no es tanto un desafío gracias al desarrollo de tecnologías, en las cuales no se profundizarán, pero si el método le resulta interesante se sugiere investigar, como "electron-shelving method".

\subsection{Superconductores}
Los superconductores son uno de los acercamientos más prometedores de la actualidad, siendo este el método elegido por algunas de las mayores empresas de tecnología en el sector, un ejemplo de esto es la computadora cuántica de Google "sycamore" y la computadora de IBM "eagle".\\
El acercamiento a crear computadoras cuánticas con superconductores es muy diferente al de los métodos anteriores, en los otros métodos trabajamos con átomos dados por la naturaleza, mientras que en este método debemos crear átomos "artificiales". Para crear esto se utiliza una tecnología llamada transmon.
\subsection{Del oscilador armónico al transmon}

Como hemos visto en los otros tipos de qbit, una buena primera aproximación es un oscilador armónico, en el caso de los superconductores podemos conseguir un oscilador harmonico cuántico utilizando circuitos LC, el cual consiste de un elemento capacitante y un inductivo, al estar creado con elementos superconductores a bajas temperaturas. Este simple circuito con frecuencia $\omega$ puede tener comportamientos cuánticos. Un resonador cuántico es, entonces, un conjunto de estos osciladores cuánticos LC, con cada modo del resonador con un oscilador con una frecuencia arbitraria bien definida.\\

\begin{figure}
    \centering
    \includegraphics[width=1\linewidth]{imagenes//superconductores/LC-circuito.png}
    \caption{Circuito para un oscilador paralelo LC (QSHO) con un inductor L en paralelo con un conductor C}
    \label{fig:enter-label}
\end{figure}

Este sistema puede ser caracterizado con dos cantidades sin dimensión $\mathbf{\phi}$ y $\mathbf{n}$, donde $\mathbf{\phi}$ representa la cuantización del flujo magnético que cruza por el inductor y $\mathbf{n}$ es la diferencia entre el número de cargas en cada placa del capacitor en unidades $2e$, siendo esto un par de Cooper. Estas cantidades tienen las relaciones canónicas de conmutación.
\begin{equation}
    [\mathbf{\phi},\mathbf{n}]=i
\end{equation}
Pudiendo así escribir el hamiltoniano para estos osciladores como:
\begin{equation}
    H=\frac{E_{L}}{2}\phi^{2}+4E_{c}n^{2}
\end{equation}
Donde $E_{L}=\frac{\hbar^{2}}{4e^{2}L}$ y $E_{c}=\frac{e^{2}}{2C}$. De manera alternativa podemos escribir el hamiltoniano, según la frecuencia $\omega$ y la impedancia $z$ del circuito, definiendo
$\omega=\frac{1}{LC}$ y $z_{0}=\sqrt{\frac{L}{C}}$, tal que:
\begin{equation}
    H=\hbar\omega\left[\frac{R_{Q}}{2z_{0}}\phi+\frac{z_{0}}{2R_{q}}n\right]
\end{equation}
Donde $R_{Q}$ es la resistencia cuántica reducida para superconductores $R_{Q}=\frac{\hbar}{4e^{2}}$\\
Lineal izando el Hamiltoniano podemos encontrar los operadores escalera $a$ y $a^{\dagger}$
\begin{equation}
    a=\sqrt{\frac{R_{q}}{2z_{0}}}\phi+i\sqrt{\frac{z_{0}}{R_{q}}}n
\end{equation}

\begin{figure}
    \centering
    \includegraphics[width=0.5\linewidth]{imagenes//superconductores/QSHO-nieveles-de-energia-equidistantes.png}
    \caption{Niveles de energía del QSHO, equidistantes con distancia $\hbar\omega_{r}$}
    \label{fig:enter-label}
\end{figure}

\subsubsubsection{Uniones de Josephson}

Los resonadores superconductores no son suficientes para codificar información cuántica, ya que los niveles de energía están separados por un espacio constante $\hbar\omega$, lo cual no permite tratar cada cambio de manera individual. La manera más fácil de agregar un elemento no lineal es utilizando uniones de Josephson.\\

\begin{figure}
    \centering
    \includegraphics[width=1\linewidth]{imagenes//superconductores/josephson-qbit.png}
    \caption{Qbit de Josephson, donde el inductor no lineal $L_{j}$(representaado por una union de Josephson en la caja naranja) deriavada por el capacitor, $C_{s}$}
    \label{fig:enter-label}
\end{figure}

Las uniones de Josephson son dos campas de superconductores con una barrera oxida entre medio. El espacio de la unión está determinada por el número de pares de Cooper que hay en cada capa superconductora, este es el observable ya definido como $\mathbf{n}$, utilizando esto el hamiltoniano de la unión se puede escribir como:
\begin{equation}
    H=-\frac{E_{J}}{2}\sum^{\infty}_{n=-\infty}(\braket{n}\ket{n+1}+\braket{n+1}\ket{n})
\end{equation}
Donde $E_{J}$ es la energía de Josephson. Los operadores $\braket{n}\ket{n+1}$ y su conjugado hermitiano, son los operadores unitarios que nos permiten cambiar el nivel de energía a un nivel más o uno menos. Esta transición es la base conjugada puede ser escrita como $e^{\pm\phi}$, haciendo esta sustitución el hamiltoniano en la  base de flujo es:
\begin{equation}
    H=\frac{E_{j}}{2}[e^{i\phi}+e^{-i\phi}]=E_{j}\cos\phi
\end{equation}
Desarrollando el coseno a segundo orden nos da una energía inductiva comúnmente conocida como inductancia de Josephson $L_{j}=\frac{\hbar^{2}}{4e^{2}E_{j}}$, esta inductancia no linear es la clave para analizar los cambios de energía de manera individual.

\begin{figure}
    \centering
    \includegraphics[width=0.75\linewidth]{imagenes//superconductores/niveles-de-energia-trasmon.png}
    \caption{Niveles de energía del Trasmon. La unión de Josephson cambia los niveles de energia cuadracticos(\ref{fig:niveles-de-energia-atomo-neutro}) en sinusoidales, lo cual permite aislar los dos niveles de energia base $\ket{0}$y $\ket{1}$. Creando un subespacio computacional.}
    \label{fig:niveles-de-energia-trasmon}
\end{figure}

\subsubsubsection{Transmon}
El resultado de la integración de la unión de Josephson es conocido como transmon, modifica la forma de la energía potencial, obteniendo así el hamiltoniano modificado
\begin{equation}
    H=4E_{c}n^{2}-E_{j}\cos\phi
\end{equation}
Con $E_{c}$ es la capacitancia total del sistema. Al ingresar esta unión de Josephson la energía deja de tomar una forma estrictamente parabólica y toma ahora una forma sinusoidal, lo que hace el espectro de la energía no degenerado, lo cual nos permite dos únicos niveles de energía, creando así nuestro qbit.

\subsubsection{Ventajas y Desventajas}

En cuanto al primer criterio los qubits con superconductores están bien caracterizados y nuestras habilidades para crearlos es bastante buena, como prueba de esto están los dos computadores mencionados en el inicio, a pesar de esto su escalabilidad aún deja que desear, la necesidad de mantener los qbit a una temperatura muy baja y su tamaño mucho mayor al compararlos con otros métodos crean un gran problema al intentar escalar a sistemas de un gran número de qbits. En cambio, el segundo criterio se cumple sin problema.\\
El mayor problema se encuentra con el tercero y quinto criterio, el gran tamaño de los trasmon y la necesidad de mantenerlos a temperaturas muy bajas, hace los sistemas muy susceptibles a ser afectados por su interacción con el ambiente haciendo su tiempo de coherencia bajo, la aplicación de compuertas de un solo qubit no representa ninguna dificultad, pero el tiempo de coherencia representa un desafío muy grande al aplicar compuertas a múltiples qubits, además de que los hamiltonianos discutidos solo se cumplen cuando el número de fotones no supera un punto crítico. Esta dificultad de limitar las interacciones con el ambiente afecta también su tiempo de lectura, donde una vez más los resultados de la aplicación de compuertas a un único qubit pueden ser leídos sin problema, pero siendo muy difíciles para compuertas de múltiples qubits. El cuarto criterio es cumplido sin ningún tipo de problema.\\

\subsection{Otros qbits}

\textbf{Fotones:} Esta tecnología está siendo desarrollada por empresas como Xanadu, PsiQuantum y otras instituciones, arrojando resultados prometedores. En los diferentes métodos para crear qbits la interacción de estos sistemas con sus alrededores crean una dificultad para la escalabilidad de estos sistemas, cosa que no ocurre utilizando fotones su interacción prácticamente inexistente con sus alrededores y ellos mismos hace que su estabilidad sea teóricamente infinita. Sin embargo, esta ventaja es a la vez un problema, ya que hace muy difícil la aplicación de compuertas a estos qbits.\\
Para manipular los fotones se utilizan principalmente "linear materials" que no interactúa con la luz. Con estos materiales podemos crear un conjunto de estados Gausianos. Para obtener los estados Gausianos deseados se utilizan operaciones Gaussianas que transforman un estado en otro. Estas operaciones son fáciles de crear utilizando elementos ópticos comunes en los laboratorios, como:
guías de onda, polarizadores de onda y expansores de haz.\\

\textbf{Spin en semiconductores:} En este método el spin de electrones o núcleos en semiconductores, principalmente en silicona, pero también se ha logrado utilizar otros semiconductores como galio, arsénico, germanio y recientemente grafeno. Los núcleos o electrones colocados en el centro de los semiconductores son expuestos a un poso de potencial, este sistema se conoce como "quantum dots", sobre los posos de potencial se coloca un electrodo de puerta del transistor, al aplicar diferentes voltajes a la compuerta se puede modificar el spin de los núcleos o electrones.\\
Las principales ventajas de este método son su capacidad de ser creados con tecnología ya existente, ya que la tecnología necesaria es la misma que la utilizada para crear semiconductores, su alto tiempo de coherencia, el hecho de que al estar rodeados por un semiconductor esto lo aislá del ambiente externo su tamaño permite crear chips con múltiples qbits. Los principales problemas de este método es el ruido causado por fluctuaciones electrónicas en la compuerta, los dos niveles de energía suelen ser muy cercano el uno del otro y el crearlos en un proceso inconsistente. Es importante mencionar que la posibilidad de escalar esta tecnología aún no ha sido probado en la práctica.

\section{Arquitectras y tecnologias referentes a la computación cuántica}

La computación cuántica promete revolucionar la tecnología al permitir el desarrollo de algoritmos capaces de realizar tareas que las computadoras clásicas no pueden ejecutar de manera eficiente. Sin  embargo, para alcanzar este futuro, es necesario crear un sistema de miles o incluso millones de qubits interconectados.\\

Ya hemos visto algunas de las tecnologías usadas para crear qubits, el diseño de la arquitectura del hardware que permita operarlos sigue siendo una de las mayores dificultades para la construcción de computadoras cuánticas escalables y funcionales. Esta información es más confidencial y difícil de acceder en comparación de los detalles sobre la creación de qubits. Por esa razón, este artículo presentará una visión general del desarrollo de la arquitectura de los diferentes chips cuánticos.\\

Es fundamental comprender que estas arquitecturas siempre combinará una parte clásica, la cual es necesaria para cosas como la corrección de error y la ejecución de algoritmos que utilicen una combinación de lógica clásica y cuántica. Además, es importante destacar que estos desarrollos tecnológicos son muy heterogéneos entre las diferentes empresas y universidades. Por lo tanto, estas notas se enfocarán en explicar las dificultades actuales asociadas a la creación de una arquitectura para una computadora cuántica.\\

Este capítulo se dividirá en dos secciones. La primera abordará las dificultades para construir computadoras cuánticas basadas en el modelo de circuitos, considerado el estándar universal en la computación cuántica. La segunda sección se centrará en la corrección de errores, un aspecto crucial debido a la fragilidad de los qubits frente a las interacciones con el entorno. En general, el capítulo se enfocará en los desafíos tecnológicos que deben superarse para desarrollar un hardware que permita construir computadoras cuánticas flexibles y escalables.

 
\subsection{Hacia una computación cuántica de pila completa(full-stack)}\label{sec2}

Como mencionado brevemente en la introducción, una computadora cuántica siempre va a consistir de una parte clásica y una cuántica, esto por dos principales razones: la aplicación de una computadora cuántica a un contexto real siempre tendrá algoritmos donde la lógica clásica mostrará un mejor resultado que la lógica cuántica; y la segunda, la introducción de error requiere un seguimiento de la parte clásica para corregir cualquier error introducido en la parte cuántica.\\

\begin{figure}[hbtp]
    \centering
    \includegraphics[width=\linewidth]{imagenes//stack/stack-quantum.png}
    \caption{Ejemplo del full stack de una computadora cuántica. tomado de a heterogeneus quantum computer}
    \label{fig:stack-qc}
\end{figure}

En la actualidad el stack que se propone para una computadora cuántica  puede ser observada en la figura \ref{fig:stack-qc}. Comenzando el análisis desde la cima de la torre podemos ver las casillas de color rojo, tenemos que en la primera capa de la pila se colocarían los algoritmos creados por el usuario, en la segunda capa, los lenguajes de programación que se desarrollarían para explotar las capacidades del hardware cuántico y los paradigmas de estos lenguajes de programación(Los paradigmas de programación son las filosofías que guían la forma en la que los desarrolladores crean software. Estos son:  imperativo, funcional, orientado a objetos, etc.), actualmente el paradigma más popular para el desarrollo de software cuántico es el orientado a objetos. Un ejemplo de esto es qiskit, la paquetería de python desarrollada por IBM o en ese mismo lenguaje de programación tenemos otros ejemplos como la paquetería de código abierto CIRq desarrollada por Google.

En la tercera capa podemos ver bloques color naranja, en estos bloques se encuentra la aritmética cuántica y el compilador, en esta parte se crearan los qubits lógicos y se compilara los algoritmos, estos compiladores constaran de dos partes una primera donde se compilara la parte clásica de los algoritmos y después una parte cuántica donde se transformara el código clásico que se desea correr en los qubits lógicos a compuertas que puedan ser ejecutados en estos.\\

La siguiente capa consta del conjunto de instrucciones de la arquitectura cuántica o por sus siglas en inglés QISA, esta capa la podemos ver caracterizada de color amarillo, esta es la capa que conecta la parte de software de la primera capa con el hardware de las últimas capas. Aquí se toma las compuertas compiladas en la capa anterior y las transforma en instrucciones que realiza el hardware, además de conectar esta información con la parte de la corrección de errores que se encuentra en la siguiente capa y se explorara en profundidad en la siguiente sección por ser esta uno de las mayores dificultades para crear una computación cuántica escalable, 


La siguiente capa, conocida como la capa de control, representada en la pila con color verde, se encarga de tomar la información procesada en el QISA. En esta etapa, el algoritmo desarrollado en un lenguaje de programación se traduce a información binaria comprensible para el sistema. Esto está basado en la tabla de Q Symbol, un ejemplo de esta tabla se puede observar en \ref{fig:Q-symbols}, en esta tabla la computadora guarda la información sobre la ubicación de todos los qubits físicos que continúan vivos, su tiempo de coherencia y las compuertas que se les han aplicado. En esta misma capa se encuentra la parte de corrección de errores, la cual se explicará en profundidad en la siguiente sección.\\

En los costados de la pila, de color gris, abarcando 3 niveles, se encuentran el nombre de algunos algoritmos del subsistema que se encargan de la corrección de error; algunos de los mostrados en la pila serán explicados de manera detallada en la siguiente sección.

\begin{figure}
    \centering
    \includegraphics[width=\linewidth]{imagenes/stack/Captura de pantalla 2024-11-26 100500.png}
    \caption{Ejemplo de una entrada en la tabla de simbolos Q.}
    \label{fig:Q-symbols}
\end{figure}

Por último, al final de la pila se encuentran los procesadores, uno clásico y el otro cuántico, donde se encuentran nuestros qubits físicos.\\


\subsection{Corrección de error.}

Como se ha venido hablando en la sección anterior una de las mayores dificultades de la computación cuántica es la fragilidad que los estados cuánticos. En qbits superconductores se tienen tiempos de coherencia no mucho mayores a más de 400 milisegundos. %citar pagina ibm %
y un porcentaje de error de alrededor de un 0.1 \% por lo que es imposible imaginar una computadora cuántica sin corrección de errores.\\

La corrección de error en una computadora cuántica varía mucho de la de una clásica y es particularmente más complicada por tres principales razones:
\begin{itemize}
    \item No podemos copiar los estados  (teorema de la no clonación).
    \item Medir un estado puede destruir la información del qbit.
    \item El error es continuo, a diferencia de computación clásica donde un bit solo puede valer 1 o 0 por lo que el único error posible es un cambio de fase, osea que en lugar de obtener el resultado correcto 1 se obtenga un 0 o viceversa, a comparación de una computadora cuántica donde el estado dado por $\psi=A\ket{i}+B\ket{0}$ puede tener una infinidad de error.
\end{itemize}

Por fortuna se ha demostrado que cualquier grupo de errores se puede descomponer como una suma de compuertas en el grupo de Pauli $\{\mathbf{1},\mathbf{X},\mathbf{Z},\mathbf{XZ}\}$. Pero por la digitalización del error se ha descubierto que son fundamentalmente dos tipos de errores los que se presentan, los causados por $\mathbf{X}$, que cambian la fase del qbit tal que $X\ket{1}=\ket{0}$ y la operación contraria $X\ket{0}=\ket{1}$los de tipo $\mathbf{Z}$ que cambian la orientación tal que $Z\ket{1}=-\ket{1}$ y $Z\ket{0}=\ket{0}$.\\

En computación clásica una de las formas más sencillas de hacer corrección de error es agregar redundancia, esto se hace copiando la información de un bit de manera que las operaciones se puede repetir en 2 o más bit para ayudar a la corrección de error. Esto no se puede replicar en un qbit por el teorema de la no clonación y el colapso de la función de onda de un qbit al medir, nosotros podemos medir la información de un bit en cualquier momento sabiendo que al medir no estamos cambiando la información, pero esto no es posible replicarse en un qbit.

Una manera de agregar redundancia en un algoritmo cuántico es expandir el espacio de Hilbert en el que están codificados los qbits. Para explicar esto veamos el código de qbit para detectar, actuando en el estado general $\ket{\psi}$ tiene la siguiente acción:

\begin{equation}
    \ket{\psi}=\lambda\ket{0}+\beta\ket{1}\xrightarrow{codificador-de-2-qbits}\ket{\psi}_{L}=\lambda\ket{00}+\beta\ket{11}=\lambda\ket{0}_{L}+\beta\ket{1}_{L}
\end{equation}

La finalidad de esto es distribuir la información del estado cuántico $\ket{\psi}$ en el estado lógico bipartito entrelazado $\ket{\psi}_{L}$, agregando así redundancia sin tener que copiar el estado.\\

Los qbits que se entrelazan son conocidos como qbits de información y los resultantes son los qbits lógicos sobre los que se operarán. Además de esto se utilizan otros qbits como refuerzo del qbit lógico para intentar medir el error, estos se le conoce como ancilla qbits o qbits esclavos. De este modo, se puede preservar la información de los qubits de datos. La medición convierte los errores continuos en errores discretos y permite identificar si hay errores y, en caso afirmativo, de qué tipo (bit- flip, fase-flip o ambos) y en qué qubit(s) se producen. Este tipo de medición se denomina medición del síndrome de error (por sus siglas en inglés ESM). \\

Es importante notar que los errores no son corregidos de manera inmediata, sino que se guardan en bits de lógica clásica utilizando una técnica llamada el marco de Pauli ("Pauli Frame"). Las operaciones y mediciones cuánticas se traducen por el marco de Pauli, preservando la corrección de la computación cuántica, como el error podemos escribirlo como una matriz X o Z de manera que necesitaremos siempre 2n bits clásicos para guardar el error, donde n es el número de qubits.\\

\subsubsection{Código de superficie}

Actualmente, la tecnología más utilizada para la corrección de errores es la llamada código de superficie o más conocida por su nombre en inglés "surface code" En el código de superficie, los qubits se organizan en un enrejado 2D regular que solo permite la interacción con los vecinos más cercano (NN). La arquitectura de NN es una de las estructuras más prometedoras, ya que el código de superficie tiene un umbral de error del 1\%, lo que significa que puede tolerar una tasa de error físico de hasta 0,01.\\

Lograr realizar un código de superficie rápido es una de las metas claves para poder tener hardware de computación cuántica efectiva. Este código forma parte de la familia de códigos topológicos. El principio general de diseño
de los códigos topológicos es que se construye uniendo elementos repetidos. Un ejemplo de esta superficie la podemos ver en la imagen \ref{fig:superficie_2d}:\\

\begin{figure}[hbtp]
    \centering
    \includegraphics[width=0.5\linewidth]{imagenes/stack/2d_surface.png}
    \caption{Ejemplo de una superficie NN}
    \label{fig:superficie_2d}
\end{figure}


\begin{figure}[hbtp]
    \centering
    \includegraphics[width=0.5\linewidth]{imagenes/stack/codigo-superficie.png}
    \caption{Traducción a código de circuito de una superficie NN}
    \label{fig:codigo-superficie}
\end{figure}

Los círculos en blanco representan qubits de datos y los círculos rellenos en verde y rojo corresponden a qbits de ancilla Z y X, respectivamente. Esto se traduce en código de circuito para cada 2 qbits lógicos como en la figura \ref{fig:codigo_superficie} donde $D_{1}$ y $D_{2}$ representan los qbits lógicos y $A_{1}$ y $A_{2}$ los qbits ancilla.

\subsubsection{Otros códigos de corrección de error.}

Mencionando de manera breve otras alternativas para la corrección del error, con la intención de que si el tema le genera interés, puede investigar más detalladamente:

\textbf{El código de corrección de error de Bacon-Shor}, es igualmente un código de subsistema que forma un enrejado 2d para almacenar el error.\\

En este método se colocan qubits físicos en los vértices. Los errores se pueden almacenar midiendo solo los vecinos más cercanos de dos qubits, utilizando los generadores estabilizadores del código de Shor. Creando el enrejado, que se muestra en la imagen.\\

\begin{figure}[hbtp]
    \centering
    \includegraphics[width=0.5\linewidth]{imagenes/stack/Captura de pantalla 2024-11-11 202625.png}
    \caption{Geometría del código Bacon-Shor(XX y ZZ son los estabilizadores, no confundir con las matrices de Pauli).}
    \label{fig:codigo-superficie}
\end{figure}

A diferencia de otros códigos de corrección de error, del subsistema, el código Bacon-Shor utiliza menos restricciones estabilizadoras, lo que simplifica las mediciones y reduce la sobrecarga en la corrección de errores. La naturaleza sub sistémica del código le permite corregir errores con menos recursos y mantener una sólida protección contra la decoherencia y el ruido cuántico.

\textbf{Código de colores(colour code):} es un tipo de código topológico de corrección de errores que codifica qubits lógicos en qubits físicos dispuestos en una red. Es una variante del código de superficie, pero utiliza un mosaico diferente de la red, normalmente basado en teselaciones triangulares o hexagonales. La principal característica del código de colores es que se basa en gráficos tricolores en los que las regiones adyacentes de la red tienen asignados distintos colores (rojo, verde y azul).\\

\begin{figure}
    \centering
    \includegraphics[width=1\linewidth]{color-code.png}
    \caption{El (4.8.8) código de colores para diferentes distancias D. Los círculos blancos representan qubtis lógicos.}
    \label{fig:enter-label}
\end{figure}

El código de colores protege la información cuántica codificándola en la topología de la red, lo que la hace resistente a errores locales. Utiliza operadores estabilizadores (combinaciones de XX y ZZ) para detectar y corregir errores.\\

El código de colores es ventajoso por su implementación de puertas tolerantes a fallos y su corrección uniforme de errores, pero tiene el coste de una mayor sobrecarga de qubits y procedimientos de medición más complejos. 

\section{Conclusión}

Durante las últimas décadas, la computación cuántica se ha comenzado a afianzar como una de las tecnologías más importantes para el futuro y hacer de ella una realidad. Como muestra de ello, está la publicación sobre la creación de la primera computadora cuántica que rompe la barrera de los mil qubits, llamada Condor y creada por IBM.\\

Espero que estas notas funcionen como una buena introducción al tema y contribuyan a su curiosidad por investigar más sobre el tema y mantenerse informado sobre los múltiples descubrimientos que ocurren cada día.


\newpage
\nocite{*}
\bibliographystyle{nature}
\bibliography{bibliography}

\end{document}


